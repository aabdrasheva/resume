\documentclass{resume} 

\usepackage[T2A]{fontenc}    
\usepackage[utf8]{inputenc}  
\usepackage[english,russian]{babel} 
\usepackage[svgnames]{xcolor}
\usepackage[left=0.4 in,top=0.4in,right=0.4 in,bottom=0.4in]{geometry} % Document margins
\newcommand{\tab}[1]{\hspace{.2667\textwidth}\rlap{#1}} 
\newcommand{\itab}[1]{\hspace{0em}\rlap{#1}}
\name{Абдрашева Асия} 
\address{+7(913) 663-9560 \\ Москва, Россия \\ Астана, Казахстан} 
\address{\href{mailto:asiyaabdrasheva@gmail.com}{asiyaabdrasheva@gmail.com} \\ \href{https://www.linkedin.com/in/asiya-abdrasheva-28b8b9212}{linkedIn} \\ \href{https://github.com/aabdrasheva}{github}}  %

\begin{document}

%----------------------------------------------------------------------------------------
%	OBJECTIVE
%----------------------------------------------------------------------------------------

\begin{rSection}{Описание}

{Бизнес аналитик (бывший бэкенд программист на Java) с опытом работы 1.5 года.}


\end{rSection}
%----------------------------------------------------------------------------------------
%	EDUCATION SECTION
%----------------------------------------------------------------------------------------

\begin{rSection}{Образование}

{\bf Бакалавр компьютерных наук},   НИУ “Высшая школа экономики”  \hfill {2019 - now}  \\
Факультет компьютерных наук, ОП "Программная инженерия" \\
GPA: 6.78 \\
{\bfКурсовые работы}: 
 \begin{itemize}
    \itemsep -3pt {} 
     \item "Анализ fullstack-платформы для компании 1С" \\\textit{Разработка модели машинного обучения, которая будет анализировать данные пользовательских сессий.}
     \item "Доработка модульной библиотеки Kmath для работы с математическими абстракциями" \\\textit{Добавление новых модулей для работы с математическими операциями.}
    \item "Андроид-приложение для сканирования авиабилетов" \\\textit{Сканирование штрихкода типа PDF417 для считывания информации о пассажире и рейсе через камеру телефона.} 
       \item "Десктопное приложение для планирования дел и задач Daily planner" \\\textit{Приложение для продуктивной работы, предназначенное для управления задачами и проектами.}
\end{itemize}



\end{rSection}

%----------------------------------------------------------------------------------------
% TECHINICAL STRENGTHS	
%----------------------------------------------------------------------------------------
\begin{rSection}{НАВЫКИ}

\begin{tabular}{ @{} >{\bfseries}l @{\hspace{6ex}} l }
Hard Skills &  \colorbox{LightGreen}{Excel}    \colorbox{LightGreen}{PowerPoint}  \colorbox{LightGreen}{Word}  \colorbox{LightGreen}{Power BI}   \colorbox{LightGreen}{Miro}    \colorbox{LightGreen}{Jira}  \colorbox{LightGreen}{Confluence}   \colorbox{LightGreen}{SQL} \colorbox{LightGreen}{Python}     
\\
   & \colorbox{LightGreen}{BPMN}   
\colorbox{LightGreen}{UML}   \colorbox{LightGreen}{ТЗ} \colorbox{LightGreen}{CRM Мегаплан}   \colorbox{LightGreen}{Java}\\

   & \\

Soft Skills & \colorbox{LightGreen}{Работа в команде}   \colorbox{LightGreen}{Тайм менеджент}    \colorbox{LightGreen}{Желание учиться}   \colorbox{LightGreen}{Умение адаптироваться}\\


\end{tabular}\\
\end{rSection}

\begin{rSection}{ОПЫТ}

\textbf{Бизнес аналитик} \hfill Декабрь 2022 - по н.в.\\
GeekBrains \hfill \textit{Москва, RU}
 \begin{itemize}
    \itemsep -3pt {} 
     \item Создание и редактирование бизнес процессов, сценариев
     \item Анализ текущих бизнес процессов
     \item Сбор и описание бизнес требований и ТЗ
     \item Тестирование CRM системы Мегаплан
     \item Администрирование CRM системы Мегаплан
 \end{itemize}
 
\textbf{Стажер Java разработчик} \hfill Апрель 2022 - Ноябрь 2022\\
Сбер \hfill \textit{Москва, RU}
 \begin{itemize}
    \itemsep -3pt {} 
     \item Разработка программного обеспечения (Intellij Idea)
     \item Bugfix
     \item Поддержка legacy-кода 
     \item Написание unit-тестов с использованием JUnit5, а также auto-тестов
     \item Написание SQL-запросов

 \end{itemize}

 \textbf{Java разработчик} \hfill Ноябрь 2021 - Апрель 2022\\
Diasoft \hfill \textit{Москва, RU}
 \begin{itemize}
    \itemsep -3pt {} 
     \item Разработка программного обеспечения (Intellij Idea)
     \item Bugfix
     \item Написание unit-тестов с использованием JUnit5
     \item Поддержка legacy-кода 
 \end{itemize}

   \textbf{Ассистент по теории вероятности и математической статистики} \hfill Сентябрь 2021 - Июнь 2022\\
Национальный исследовательский университет “Высшая школа экономики” \hfill \textit{Москва, RU}
 \begin{itemize}
    \itemsep -3pt {} 
     \item Проверка домашних работ, которые студенты сдают каждые 2–4 недели
     \item Проведение консультаций по необходимости студентов
     \item Помощь преподавателям: лектору и семинаристу 
 \end{itemize}

\end{rSection} 

%----------------------------------------------------------------------------------------
%	WORK EXPERIENCE SECTION
%----------------------------------------------------------------------------------------

\begin{rSection}{ПРОЕКТЫ}
\vspace{-1.25em}
\item \textbf{Приложение для построения фигур и вычисления площади сечения.} {Я создала десктопное приложение для визуализации геометрических фигур с возможностью задать плоскость сечения с помошью 3 точек / прямой линии и точки, не лежащая на этой прямой / двух пересекающихся прямых / двух параллельных прямых с вычислением площади данного сечения. Приложение написано на С Sharp.}
\item \textbf{Portfolio Website.} {В процессе создание сайта-резюме с использованием HTML, CSS и JavaScript.}
\end{rSection} 


\end{document}
